% !Mode:: "TeX:UTF-8"

%%  可通过增加或减少 setup/format.tex中的
%%  第247行 \setlength{\@title@width}{5.5cm}中 5.5cm 这个参数来 控制封面中下划线的长度。

\cheading{天津大学~2016~届本科生毕业论文}      % 设置正文的页眉,需要填上对应的毕业年份
\ctitle{基于声管的语音合成}    % 封面用论文标题,自己可手动断行
\caffil{计算机科学与技术学院} % 学院名称
\csubject{计算机科学与技术}   % 专业名称
\cgrade{2012~级}            % 年级
\cauthor{张洁}            % 学生姓名
\cnumber{3012216058}        % 学生学号
\csupervisor{王建荣}        % 导师姓名
\crank{副教授}              % 导师职称

\cdate{\the\year~年~\the\month~月~\the\day~日}

\cabstract{
基于声管的语音合成是基于发音机理的语音合成方法的重要组成部分。本研究将基于核磁共振(MRI)数据,采用时域模拟方法,用传输线电路TLM来模拟声道,并加入了噪声源模型。模型中,控制声波生成和传播的声波方程通过应用一定的规则转化为离散变量,并在基于一个更现实的对流体动压变化的分布式考虑及基础上进行改进,同时考虑声道的分支将三个不同稀疏矩阵运用数学方法合并成单一矩阵,以此来完善现有的元音的声管模型,使模型能更准确的生成元音、辅音,并且可以成功避免声伪像。
}

\ckeywords{发音语音合成;声道;时域模拟方法;噪声源}

\eabstract{
Articulatory synthesis based on acoustic pipe is an important part of the articulatory synthesis method which is based on the articulatory mechanism. This research will be based on magnetic resonance imaging (MRI) data, time domain simulation method, using electrical transmission-line(TLM) circuit to simulate the vocal tract and joined the noise sources model. The acoustic equations that govern the generation and the propagation of acoustic waves inside the model were transformed into the discrete variable representation by applying certain rules. We improve the model basing on a more realistic distributed consideration of fluid dynamic pressure changes. At the same time, considering vocal tract branch, we will merge three different sparse matrixes into a single matrix by using mathematical methods. All above is to improve the existing vowel sound tube model and make the model generating vowels and consonants more accurately and avoiding artifacts successfully.
}

\ekeywords{Articulatory synthesis; Vocal tract; Time domain simulation; Noise Sources}

\makecover

\clearpage
